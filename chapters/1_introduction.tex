% !TeX encoding = UTF-8
% !TeX spellcheck = en_GB
% !TeX root = ../thesis.tex


\chapter{Introduction}
\label{chapter:introduction}

To start using this template, change the values inside \texttt{thesis.tex} to your liking. \emph{It is strongly advised to change what is necessary (language, names, etc.).} Please also remember to remove the "REMOVE ME LATER" part when you are done. Below, you can find some examples of important \LaTeX-commands. The chapters and sections are more of a suggestion of what the thesis should contain. Please read the books and papers mentioned in the references \cite{DBLP:books/sp/GrubaZ17, DBLP:books/sp/EvansGZ14, DBLP:books/sp/Zobel14, DBLP:conf/cikm/Zobel15, DBLP:conf/icse-seeng/MauererKS22}.

This is a cited text~\cite{DBLP:books/sp/Zobel14}.

\todo{This is an example of the todo-command.}

This refers to \Cref{chapter:discussion,sec:future-work}.

We can also add code such as in \Cref{lst:log}.

\begin{code}
    \captionof{listing}{Captions for listings are usually placed above. Java code calculates the logarithm of a given number concerning a given base.}
    \label{lst:log}
    \begin{minted}[frame=lines,framesep=2mm,baselinestretch=1.2,fontsize=\footnotesize,linenos]{java}
public static double log(double x, double base) {
    return Math.log(x) / Math.log(base);
}
    \end{minted}
\end{code}

You can also include and reference figures and tables such as \Cref{fig:test} and \Cref{tab:test}. Notice how the table cannot be placed using the "h" specifier and thus uses the "t" specifier instead?

\begin{figure}[ht!]
    \centering
    \includegraphics[width=0.5\textwidth]{img/logouni_en.png}
    \caption{Captions for figures are usually placed below. The German logo of the University of Passau.}
    \label{fig:test}
\end{figure}

\clearpage

URLs can be added like this:

\begin{itemize}
    \item Find better bibliography entries at \url{https://dblp.org}
    \item Organize literature with Zotero\footnote{\url{https://www.zotero.org}}
    \item Create drawings and presentations with IPE\footnote{\url{https://ipe.otfried.org/}}
    \item IPE Tutorial\footnote{\url{https://i11www.iti.kit.edu/information/ipe_tutorial}}
    \item Paper about Reproducibility Engineering \cite{DBLP:conf/icse-seeng/MauererKS22}\footnote{\url{https://www.lfdr.de/Publications/2022/SEENG-Mauerer.pdf}}
\end{itemize}

\begin{table}[ht!]
    \centering
    \caption{Captions for tables are usually placed above. $\phi$ denotes the Euler totient function, by the way.}
    \label{tab:test}
    \begin{tabular}{c c}
        \hline
        $x$ & $\phi(x)$ \\
        \hline
        $1$ & $1$ \\
        $2$ & $1$ \\
        $3$ & $2$ \\
        $4$ & $2$ \\
        $5$ & $4$ \\
        $6$ & $2$ \\
        \hline
    \end{tabular}
\end{table}

Or add algorithms \Cref{alg:two}.

\begin{algorithm}
    \caption{An algorithm with caption. \cite{noauthor_algorithms_nodate}}
    \label{alg:two}
    \KwData{$n \geq 0$} 
    \KwResult{$y = x^n$}
    $y \gets 1$,
    $X \gets x$,
    $N \gets n$\;
    \While{$N \neq 0$}{
      \eIf{$N$ is even}{
        $X \gets X \times X$\;
        $N \gets \frac{N}{2}$ \Comment*[r]{This is a comment}
      }{\If{$N$ is odd}{
          $y \gets y \times X$\;
          $N \gets N - 1$\;
        }
      }
    }
\end{algorithm}

\section{Motivation}
\label{sec:motivation}

\lipsum
%\lipsum[3-56]

\section{Problem Statement}
\label{sec:problem-statement}

\lipsum

\section{Objectives}
\label{sec:objectives}

\lipsum

\section{Outline}
\label{sec:outline}

\lipsum